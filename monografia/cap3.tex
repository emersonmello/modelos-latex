% ----------------------------------------------------------------------- %
% Arquivo: cap3.tex
% ----------------------------------------------------------------------- %
\chapter{Conceitos finais sobre o documento}
\label{c_cap3}

Neste capítulo, diferentemente do ocorreu na \autoref{s_c2_figuras} do \autoref{c_cap2}, será apresentado uma forma para inserir tabelas no documento. A \autoref{t_c3_etapas} é só um pequeno exemplo de tabela.

\begin{table}[!htpb]
\ABNTEXfontereduzida
\centering
\caption{Cronograma das atividades previstas}
\label{t_c3_etapas}
 \setlength{\tabcolsep}{3pt}
\begin{tabular}{|c|c|c|c|c|c|c|c|c|c|c|c|c|}\hline
 & \multicolumn{12}{c|}{Semanas}\\ \cline{2-13}
\raisebox{1.5ex}{Etapa} & 01 & 02 & 03 & 04 & 05 & 06 & 07 & 08 & 09 & 10 & 11 & 12 \\ \hline
1 & $\surd$ & $\surd$ & $\surd$ & & & & & & & & & \\ \hline
2 & & & & $\surd$ & $\surd$ & $\surd$ & $\surd$ & & & & & \\ \hline
3 & & & & & & & & $\surd$ & $\surd$ & $\surd$ & & \\ \hline
4 & & & & & & & & & & & $\surd$ & $\surd$ \\ \hline
\end{tabular} 
\end{table} 

\index{tabelas}A \autoref{tab-nivinv} é um exemplo de tabela construída em \LaTeX.

\begin{table}[htb]
\ABNTEXfontereduzida
\caption[Níveis de investigação]{Níveis de investigação.}
\label{tab-nivinv}
\begin{tabular}{p{2.6cm}|p{6.0cm}|p{2.25cm}|p{3.40cm}}
  %\hline
   \textbf{Nível de Investigação} & \textbf{Insumos}  & \textbf{Sistemas de Investigação}  & \textbf{Produtos}  \\
    \hline
    Meta-nível & Filosofia\index{filosofia} da Ciência  & Epistemologia &
    Paradigma  \\
    \hline
    Nível do objeto & Paradigmas do metanível e evidências do nível inferior &
    Ciência  & Teorias e modelos \\
    \hline
    Nível inferior & Modelos e métodos do nível do objeto e problemas do nível inferior & Prática & Solução de problemas  \\
   % \hline
\end{tabular}
\legend{Fonte: \citeonline{van86}}
\end{table}

Já a \autoref{tabela-ibge} apresenta uma tabela criada conforme o padrão do \citeonline{ibge1993} requerido pelas normas da ABNT para documentos técnicos e acadêmicos.

\begin{table}[htb]
\IBGEtab{%
  \caption{Um Exemplo de tabela alinhada que pode ser longa
  ou curta, conforme padrão IBGE.}%
  \label{tabela-ibge}
}{%
  \begin{tabular}{ccc}
  \toprule
   Nome & Nascimento & Documento \\
  \midrule \midrule
   Maria da Silva & 11/11/1111 & 111.111.111-11 \\
  \midrule 
   João Souza & 11/11/2111 & 211.111.111-11 \\
  \midrule 
   Laura Vicuña & 05/04/1891 & 3111.111.111-11 \\
  \bottomrule
\end{tabular}%
}{%
  \fonte{Produzido pelos autores.}%
  \nota{Esta é uma nota, que diz que os dados são baseados na
  regressão linear.}%
  \nota[Anotações]{Uma anotação adicional, que pode ser seguida de várias
  outras.}%
  }
\end{table}



\section{Como usar referências bibliográficas}
\label{s_c3_referencias}

O uso de citações ao londo do texto é uma prática desejável. Por exemplo, em \cite{lamport94} é apresentado um documento sobre a preparação de textos usando \LaTeX. Já em \cite{goossens94} é apresentada uma lista de referências rápidas para realizar as mais simples tarefas em \LaTeX.

É o caso em que você menciona \emph{explicitamente} o autor da referência na sentença, algo
do tipo ``Fulano (1900)''. Neste caso o nome do autor é escrito
normalmente. Para isso use o comando \verb+\citeonline+.

A ironia será assim uma \ldots\ proposta  por \citeonline{lamport94}. Em \cite{exemplo} foi usado para ilustrar como uma \textit{URL} deve aparecer na seção das referências. Este documento segue as normas da \ac{ABNT} e para isso faz uso do pacote \ac{abnTeX}.


% ---
\section{Citações diretas}
\label{sec-citacao}
% ---

\index{citações!diretas}Utilize o ambiente \texttt{citacao} para incluir
citações diretas com mais de três linhas:

\begin{citacao}
As citações diretas, no texto, com mais de três linhas, devem ser
destacadas com recuo de 4 cm da margem esquerda, com letra menor que a do texto
utilizado e sem as aspas. No caso de documentos datilografados, deve-se
observar apenas o recuo \cite[5.3]{NBR10520:2002}.
\end{citacao}

Use o ambiente assim:

\begin{verbatim}
\begin{citacao}
As citações diretas, no texto, com mais de três linhas [...] deve-se observar
apenas o recuo \cite[5.3]{NBR10520:2002}.
\end{citacao}
\end{verbatim}

O ambiente \texttt{citacao} pode receber como parâmetro opcional um nome de
idioma previamente carregado nas opções da classe. Nesse
caso, o texto da citação é automaticamente escrito em itálico e a hifenização é
ajustada para o idioma selecionado na opção do ambiente. Por exemplo:

\begin{verbatim}
\begin{citacao}[english]
Text in English language in italic with correct hyphenation.
\end{citacao}
\end{verbatim}

Tem como resultado:

\begin{citacao}[english]
Text in English language in italic with correct hyphenation.
\end{citacao}

\index{citações!simples}Citações simples, com até três linhas, devem ser
incluídas com aspas. Observe que em \LaTeX as aspas iniciais são diferentes das
finais: ``Amor é fogo que arde sem se ver''.






