% 2019-05-15 Emerson Ribeiro de Mello - mello@ifsc.edu.br

\documentclass[11pt,twoside,a4paper]{inifsc}

\usepackage[brazilian]{babel}
\usepackage[utf8]{inputenc}
\usepackage{url,graphicx,pgf}
\urlstyle{sf}

%-----------------------------------------------------------
% Informações que aparecerão no cabeçalho
%-----------------------------------------------------------
% Logo
% \pgfdeclareimage[width=5.5cm]{logo}{imagens/ifsc-h-sje.png}
\pgfdeclareimage[width=1.27\linewidth]{logo}{imagens/ifsc-header.png}

% \instituicao{Ministério da Educação\\
% Secretaria de Educação Profissional e Tecnológica\\
% Instituto Federal de Santa Catarina\\ Campus São José}

%-----------------------------------------------------------
% Início do documento
%-----------------------------------------------------------
\begin{document}

%-----------------------------------------------------------
% Comandos opcionais
%-----------------------------------------------------------
% Esse comando é opcional. Pode comentá-lo
\disposicao{Dispõe de como poderia ser um modelo em \LaTeX~para escrita de regulamentos e normativas no Instituto Federal de Santa Catarina.}
    
% Esse comando é opcional. Pode comentá-lo
\assinatura{Emerson Ribeiro de Mello}{Primeiro autor do modelo}

% Esse comando é opcional. Pode comentá-lo
\aprovado{Aprovado na reunião do órgão sem representação em 15 de maio de 2019.}
%-----------------------------------------------------------


%-----------------------------------------------------------
% Início da normativa. Deve-se colocar o título da normativa
% como parâmetro do ambiente normativa
%-----------------------------------------------------------
\begin{normativa}{Modelo \LaTeX~para Regulamentos e Instruções Normativas}

%-----------------------------------------------------------
%  o agrupamento de artigos poderá constituir Seção; o de Seções, o Capítulo.
%-----------------------------------------------------------
\chapter{Disposições preliminares}

\begin{artigo}
    \item A elaboração, a redação, a alteração e a consolidação das leis obedecerão ao disposto na lei complementar Nº 95, de 26 de fevereiro de 1998\footnote{\url{http://www.planalto.gov.br/ccivil_03/leis/lcp/lcp95.htm}}.
    \begin{artigo}
        \item As disposições desta Lei Complementar aplicam-se, ainda, às medidas provisórias e demais atos normativos referidos no art. 59 da Constituição Federal, bem como, no que couber, aos decretos e aos demais atos de regulamentação expedidos por órgãos do Poder Executivo.
    \end{artigo}
\end{artigo}


\chapter{Das Técnicas de Elaboração e Redação}


\section{Da Articulação e da Redação das Leis}


\begin{artigo}
    \item\label{art:observancia}
    Os textos legais serão articulados com observância dos seguintes princípios:
    \begin{inciso}
        \item\label{inc:formatoartigo}
         a unidade básica de articulação será o artigo, indicado pela abreviatura ``Art.'', seguida de numeração ordinal até o nono e cardinal a partir deste;
        \item os artigos desdobrar-se-ão em parágrafos ou em incisos; os parágrafos em incisos, os incisos em alíneas e as alíneas em itens;
        \item os parágrafos serão representados pelo sinal gráfico §, seguido de numeração ordinal até o nono e cardinal a partir deste, utilizando-se, quando existente apenas um, a expressão ``parágrafo único'' por extenso;
        \item os incisos serão representados por algarismos romanos, as alíneas por letras minúsculas e os itens por algarismos arábicos;
        \item o agrupamento de artigos poderá constituir Subseções; o de Subseções, a Seção; o de Seções, o Capítulo; o de Capítulos, o Título; o de Títulos, o Livro e o de Livros, a Parte;
    \end{inciso}

    \item As disposições normativas serão redigidas com clareza, precisão e ordem lógica, observadas, para esse propósito, as seguintes normas: 
    \begin{inciso}
        \item para a obtenção de clareza:
        \begin{inciso}
            \item usar as palavras e as expressões em seu sentido comum, salvo quando a norma versar sobre assunto técnico, hipótese em que se empregará a nomenclatura própria da área em que se esteja legislando;
            \item usar frases curtas e concisas;
            \item construir as orações na ordem direta, evitando preciosismo, neologismo e adjetivações dispensáveis;
        \end{inciso}
        \item para a obtenção de precisão:
        \begin{inciso}
            \item evitar o emprego de expressão ou palavra que confira duplo sentido ao texto;
            \item grafar por extenso quaisquer referências a números e percentuais, exceto data, número de lei e nos casos em que houver prejuízo para a compreensão do texto; 
            \begin{inciso}
                \item Só para mostrar como seria o último nível de hierarquia
            \end{inciso}
        \end{inciso}
        \item para a obtenção de ordem lógica:
        \begin{inciso}
            \item reunir sob as categorias de agregação -- subseção, seção, capítulo, título e livro -- apenas as disposições relacionadas com o objeto da lei;
            \item restringir o conteúdo de cada artigo da lei a um único assunto ou princípio;
            \item expressar por meio dos parágrafos os aspectos complementares à norma enunciada no caput do artigo e as exceções à regra por este estabelecida;
            \item promover as discriminações e enumerações por meio dos incisos, alíneas e itens.
        \end{inciso}
    \end{inciso}
\end{artigo}



\chapter{Dos ambientes desse modelo}

\section{Dos desdobramentos}

\begin{artigo}
    \item O ambiente \textbf{artigo} possui cinco níveis de profundidade e dentro desse ambiente pode-se ainda colocar o ambiente \textbf{parágrafo} ou o ambiente \textbf{inciso}.
    \begin{artigo}
        \item Se o artigo tiver somente um parágrafo, então crie um segundo nível do ambiente artigo, como é o que está sendo feito nesse parágrafo.
        \begin{artigo}
            \item Se o artigo possuir mais de um parágrafo, então o dentro do ambiente \texttt{artigo} deve-se colocar um ambiente \texttt{parágrafo}.
        \end{artigo}
    \end{artigo}


    \item\label{art:comparagrafoaserref}
    O ambiente \textbf{parágrafo} tem quatro níveis de profundidade
    \begin{paragrafo}
        \item\label{par:nivelparagrafo}
        o primeiro nível é para representar o parágrafo
        \begin{paragrafo}
            \item\label{inc:segnivelparagrafo}
             o segundo nível é para representar o inciso
            \begin{paragrafo}
                \item\label{ali:ternivelparagrafo}
                o terceiro nível é para representar a alínea
                \begin{paragrafo}
                    \item o quarto nível é para representar o item
                \end{paragrafo}
            \end{paragrafo}
        \end{paragrafo}
    \end{paragrafo}

    \item O ambiente \textbf{inciso} tem três níveis de profundidade
    \begin{inciso}
        \item o primeiro nível é para representar o inciso
        \begin{inciso}
            \item o segundo nível é para representar a alínea
            \begin{inciso}
                \item o terceiro nível é para representar o item
            \end{inciso}
        \end{inciso}
    \end{inciso}
\end{artigo}


\section{Das referências cruzadas}

\begin{artigo}
    \item Deverá ter um \texttt{label} definido Todo artigo, parágrafo, inciso, alínea ou item que precisar ser referenciado.
    \begin{paragrafo}
        \item Aqui tem-se um exemplo de referência ao~\ref{art:observancia}.
        \item Aqui tem-se um exemplo de referência ao~\ref{art:comparagrafoaserref},~\ref{par:nivelparagrafo}.
        \item Aqui tem-se um exemplo de referência na ordem decrescente ao~\ref{art:comparagrafoaserref},~\ref{par:nivelparagrafo},~\ref{inc:segnivelparagrafo},~\ref{ali:ternivelparagrafo} do Modelo de Regimento.
        \begin{paragrafo}
            \item Aqui tem-se um exemplo de referência na ordem crescente a alínea~\ref{ali:ternivelparagrafo} do inciso~\ref{inc:segnivelparagrafo} do~\ref{par:nivelparagrafo} do~\ref{art:comparagrafoaserref} do Modelo de Regimento.    
        \end{paragrafo}
    \end{paragrafo}
\end{artigo}


\section{Das limitações conhecidas}

\begin{artigo}
    \item De acordo com o inciso V do art. 10 da lei complementar nº 95 de 26 de fevereiro de 1998, o agrupamento de artigos poderá constituir Subseções; o de Subseções, a Seção; o de Seções, o Capítulo; o de Capítulos, o Título; o de Títulos, o Livro e o de Livros, a Parte;
    \begin{artigo}
        \item Esse modelo redefiniu a apresentação das divisões \texttt{chapter} e \texttt{section}.
    \end{artigo}

    \item O ambiente \textbf{artigo} cria uma lista de itens com vários níveis de profundidade, porém seu uso deveria ser restringido somente para quando quiser criar parágrafo único.
    \begin{artigo}
        \item Seria interessante criar uma solução de listas que se houver somente um único item, esse seria numerado com parágrafo único, senão todos os itens deveriam ser numerados com o símbolo § seguido de um número cardinal.
        \begin{artigo}
            \item Enquanto isso, não use mais de um de nível de profundidade para \textbf{artigo} se você tiver mais de um parágrafo dentro desse artigo.
        \end{artigo}
    \end{artigo}

    \item É possível que existam outras limitações nesse modelo e os autores incentivam que, caso encontre alguma, reporte-a aos autores.
    
\end{artigo}

\end{normativa}
\end{document}