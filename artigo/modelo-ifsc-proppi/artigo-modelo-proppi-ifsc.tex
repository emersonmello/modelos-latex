%---------- ATENCAO -----------%
% documento codificado em UTF-8
% 2010-06-19 Emerson R. de Mello mello@ifsc.edu.br
%---------- ATENCAO -----------%

 \documentclass[12pt]{article}
% Comente a linha cima e descomente a linha abaixo para alterar para 2 colunas
%\documentclass[10pt,twocolumn]{article}

\usepackage{estilo-artigo}


% Comente a linha abaixo para espaçamento simples entre linhas
% \onehalfspacing 

% Descomente a linha abaixo para diminuir o tamanho da fonte das section, subsection e subsubsection
\diminuirTituloSecoes


\title{\textbf{Modelo de artigo para Publicações Acadêmico-científicas do IFSC}}

\author{\textbf{Primeiro Autor}\\
Telecomunicações, Instituto Federal de Santa Catarina\\
\url{email@ifsc.edu.br}
\vspace*{.2cm}\\
\textbf{Segundo Autor}\\
Doutor, Telecomunicações, Instituto Federal de Santa Catarina\\
\url{segundo@sj.ifsc.edu.br}
}

\date{}



\begin{document}

% configuracao das lengendas
\captionsetup{font=small,labelfont=bf,textfont=bf}



\maketitle

\begin{resumo}
A proposta deste modelo de artigo é servir de base para a estrutura e a formatação de artigos acadêmico-científicos a serem publicados nos periódicos do Instituto Federal de Educação, Ciência e Tecnologia de Santa Catarina - IFSC. Os artigos submetidos devem ser elaborados em Português e devem ser produto de atividades de ensino, pesquisa ou extensão. Este modelo em \LaTeX, atualizado em março de 2017, foi baseado no modelo disponibilizado pela Pró-reitoria de Pesquisa, Pós-gradução e Inovação (PROPPI).
\end{resumo}

 \begin{palavraschave}
 Palavra chave 1. Palavra chave 2. Palavra chave 3.
 \end{palavraschave}

\begin{abstract}
The purpose of this template is to be a model for the structure and the formatting of academic and scientific articles to be published in the journals of the Federal Institute of Education, Science and Technology of Santa Catarina - IFSC. Submitted papers should be prepared in Portuguese and must be the product of teaching, research or extension.
\end{abstract}

 \begin{keywords}
 keyword 1. keyword 2. keyword 3.
 \end{keywords}


\thispagestyle{empty}

\section{Introdução}
\label{s_introducao}

Este modelo em \LaTeX, atualizado em março de 2017, foi baseado no modelo disponibilizado\footnote{\url{http://www.ifsc.edu.br/images/stories/file/PRPPG/modelo_de_artigo_cientifico.odt}} pela Pró-reitoria de Pesquisa, Pós-gradução e Inovação (PROPPI).

A proposta dos periódicos de Publicações Acadêmicas do IF-SC é de publicar trabalhos em forma de artigos de caráter teórico ou aplicado, de pesquisas científicas e tecnológicas, como relato de experiências, ensaios fundamentados ou artigos de revisão.

Artigo anteriormente publicado em congressos ou conferências, e aceito para publicação em periódico do IF-SC, deverá contemplar o evento como nota de rodapé na página do título. 

A decisão de aceite para publicação será baseada na recomendação de no mínimo dois avaliadores e, se necessário, de um membro do conselho editorial. Somente os trabalhos aprovados serão encaminhados para publicação, quando serão publicados integralmente. O(s) autor(es) deve(em) manter seu arquivo para eventuais modificações sugeridas pelos revisores, visto que os originais enviados não serão devolvidos. 

\section{Submissão}

Para o envio das propostas, o candidato deverá preencher o formulário de Inscrição anexo ao edital disponível na Internet, junto ao qual deverá ser enviada uma cópia do trabalho em formato \texttt{.pdf} sem identificação de autoria ou Instituição. 

As propostas deverão ser enviadas via formulário eletrônico disponível no site do IF-SC: \url{www.ifsc.edu.br} no link: Publicações do IF-SC\footnote{Esta seção é apenas um exemplo, desconsidere esse processo de submissão.}.


\section{Edição do texto}
O artigo deverá conter no mínimo 5 (cinco) e no máximo 15 (quinze) páginas não numeradas, incluindo tabelas, quadros e figuras, e ser apresentado em uma coluna. A fonte deverá ser Arial, tamanho 11, para os títulos dos itens, dos subitens, do resumo, do texto e das referências. Não deverão existir no texto palavras em negrito ou sublinhado para destacar segmentos do texto; somente itálico.  

O espaçamento deverá ser 1,5 no corpo do texto e duplo entre itens e subitens. E o parágrafo deverá ter um centímetro de recuo na primeira linha.

O formato do papel deverá ser A4, orientação retrato, com margens espelho, nas seguintes dimensões:

\begin{itemize}
	\item interna: 3,5cm;
	\item superior e inferior: 1,27cm;
	\item externa: 2,0cm.
\end{itemize}

Os itens e subitens deverão ser alinhados à esquerda, enumerados, em negrito e com apenas a primeira letra maiúscula. Não se utilizam ponto, hífen, travessão ou qualquer outro sinal após o indicativo numérico do item ou subitem.  

As grandezas deverão ser expressas no Sistema Internacional (SI), e a terminologia científica (incluindo a nomenclatura e os símbolos gregos) deverá seguir as convenções internacionais de cada área em questão.  

\section{Composição sequencial do artigo}

O título do artigo deverá ser apresentado em negrito, na fonte Arial, tamanho 22, centralizado, com no máximo 15 palavras, somente com a primeira letra da primeira palavra maiúscula. 

O nome dos autores deve ser apresentado por extenso, em negrito, em fonte Arial, tamanho 12, com somente a primeira letra do nome e dos sobrenomes maiúscula. Deve ser seguido da titulação acadêmica, da Instituição a que está vinculado e do e-mail, em fonte Arial, tamanho 10. 

O resumo, apresentado em um único parágrafo, não deverá ter mais que 250 palavras, descrevendo os objetivos, a metodologia usada e os principais resultados e conclusões. Não deverá conter fórmulas e deduções matemáticas.  

As palavras-chave, no mínimo três e no máximo cinco, deverão representar o conteúdo do texto, iniciar com letra maiúscula e ser separadas por ponto final.
Todos os símbolos e siglas deverão ser descritos no texto na primeira vez em que forem apresentados. Cada símbolo deverá estar dimensionalmente definido no SI com unidades mencionadas. Variáveis adimensionais e coeficientes devem ser definidos e indicados. 

A introdução deverá conter informações direcionadas a todos os leitores da revista e não somente a especialistas da área. Deverá descrever o estado da arte do problema, sua relevância, resultados significativos, conclusões a partir de trabalhos anteriores e os objetivos do presente trabalho. 

Na sequência, o texto deve contemplar, além de fundamentos teóricos, Materiais e métodos (metodologia); Resultados e discussões, quando houver; Considerações finais, baseando-se nos objetivos da pesquisa, Referências e Agradecimentos se houver. 

\section{Equações matemáticas}

As equações deverão ser indicadas em um novo parágrafo. Quando necessário, deve-se utilizar toda a extensão da largura da página para edição da mesma.

As equações devem ser numeradas sequencialmente e identificadas por números arábicos entre parênteses, alinhados à direita, com a indicação de letra maiúscula. 

A referência à equação deverá ser feita, no corpo do texto, da forma abreviada; no início da frase, por extenso. Exemplo.. substituindo-se a Equação \ref{e_c2_eq1} tem-se a seguinte expressão: \ldots.;: A Equação \ref{e_c2_eq1} deverá estabelecer a relação \ldots.


\begin{equation}
 x=\frac{-b\pm\sqrt{b^2-4ac}}{2a}
\label{e_c2_eq1}
\end{equation}

\section{Figuras, tabelas e quadros}

As figuras, as tabelas e os quadros deverão ser referenciados no texto em ordem consecutiva e identificados, em negrito, por número arábico precedido da palavra correspondente (Figura \ref{f_mascote}, Tabela \ref{t_linguagens}), seguido de respectiva legenda, figurando o mais próximo possível do texto em que foram referenciados, separados dele por uma linha em branco. 

As figuras, os quadros e a sua legenda devem ser centralizados na extensão da largura da página (Figura \ref{f_mascote}). A identificação das figuras e dos quadros deve ser na parte inferior; das tabelas, na parte superior, alinhada à margem esquerda. A referência da fonte, quando não de autoria própria, deve ser colocada logo abaixo da figura, tabela ou quadro, em letra maiúscula/minúscula, precedida da palavra FONTE. As anotações e as numerações devem ser formatadas em fonte Arial, tamanho 10, e todas as unidades devem ser expressas no sistema S.I. (métrico).

\begin{figure}[!htpb]
	\centering
	\includegraphics[width=.4\columnwidth]{figs/lion}
	\caption{Mascote do \LaTeX}
	\label{f_mascote}
\end{figure}

As referências às figuras, às tabelas e aos quadros seguem o mesmo padrão das equações, referenciadas por Fig. (ou Tab. ) no corpo do texto ou por Figura \ref{f_mascote} (ou Tabela \ref{t_linguagens}) quando usada no início de uma sentença. 

As figuras que apresentam dados técnicos de resultados deverão apresentar um contorno sobre todos os quatro lados, com eixos e escalas indicadas. E as legendas para os símbolos usados nas figuras deverão ser colocadas dentro do quadro, assim como também a identificação de cada curva. Os contornos deverão ser legíveis o suficiente para evitar qualquer dúvida. 
As tabelas e os quadros, de preferência, deverão estar em preto e branco. As legendas das figuras, dos quadros e das tabelas não devem exceder três linhas. A segunda e a terceira linhas tem recuos, como mostrado na legenda da Figura \ref{t_linguagens}.

\begin{table}[!htbp]
\caption{Linguagens de programação}
\label{t_linguagens}
\begin{center}
\begin{tabular}{|l|l|}  \hline
Linguagem & Paradgima \\ \hline \hline
Pascal & Estruturada \\ \hline
C & Estruturada \\ \hline
Java & Orientada a objetos \\ \hline
PHP & Orientada a objetos \\ \hline
Python & Orientada a objetos \\ \hline
\end{tabular}
\end{center}
\end{table}


Anotações e valores numéricos nela incluídos devem ter tamanhos compatíveis com o da fonte usada no texto do trabalho, e todas as unidades devem ser expressas no sistema S.I. (métrico). As unidades são incluídas apenas nas primeiras linha/coluna, conforme for apropriado. As tabelas devem ser colocadas tão perto quanto possível de sua primeira citação no texto. Deixa-se uma linha simples em branco entre a tabela, seu título e o texto. O estilo de borda da tabela é simples de 1 pt. 

\section{Citações e referências}

As referências deverão ser citadas no texto pelo último nome do(s) autor(es), ano de publicação da obra e número da(s) página(s) (este obrigatório no caso de citação direta).  

Em \cite{lamport94} é apresentado um documento sobre a preparação de textos usando \LaTeX. Já em  \cite{goossens94} é apresentada uma lista de referências rápidas para realizar as mais simples tarefas em \LaTeX. É o caso em que você menciona \emph{explicitamente} o autor da referência na sentenca, algo  do tipo ``Fulano (1900)''. Neste caso o nome do autor é escrito normalmente. Para isso use o comando \verb+\citeonline+. A ironia será assim uma \ldots\ proposta  por \citeonline{lamport94}. \citeonline[pág.45]{lamport94} é uma outra forma.

Como regra geral, deverá ser consultada a norma da ABNT - NBR 10520 – Informação e documentação - Citações em documentos - Apresentação. As referências deverão ser listadas em ordem alfabética de autor e título para todo tipo de documento consultado e utilizado.




\bibliography{referencias}

\end{document}
