% Os exemplos daqui foram extraídos da página https://en.wikibooks.org/wiki/LaTeX/Glossary e também de um documento de exemplo do pacote abnTeX2

% exemplo de acrônimos
\newacronym[longplural={\textit{Frames per Second}}]{fps}{FPS}{\textit{Frame per Second}}


\newglossaryentry{Linux}
{
  name=Linux,
  description={is a generic term referring to the family of Unix-like computer operating systems that use the Linux kernel},
  plural=Linuces
}

\newglossaryentry{pi}
{
  name={\ensuremath{\pi}},
  description={ratio of circumference of circle to its
               diameter},
  sort=pi
}

\newglossaryentry{numero real}
{
  name={número real},
  description={include both rational numbers, such as $42$ and 
               $\frac{-23}{129}$, and irrational numbers, 
               such as $\pi$ and the square root of two; or,
               a real number can be given by an infinite decimal
               representation, such as $2.4871773339\ldots$ where
               the digits continue in some way; or, the real
               numbers may be thought of as points on an infinitely
               long number line},
  symbol={\ensuremath{\mathbb{R}}}
}

\newglossaryentry{pai}{
                name={pai},
                plural={pai},
                description={este é uma entrada pai, que possui outras
                subentradas.} }

 \newglossaryentry{componente}{
                name={componente},
                plural={componentes},
                parent=pai,
                description={descrição da entrada componente.} }
 
 \newglossaryentry{filho}{
                name={filho},
                plural={filhos},
                parent=pai,
                description={isto é uma entrada filha da entrada de nome
                \gls{pai}. Trata-se de uma entrada irmã da entrada
                \gls{componente}.} }
 
\newglossaryentry{equilibrio}{
                name={equilíbrio da configuração},
                see=[veja também]{componente},
                description={consistência entre os \glspl{componente}}
                }

\newglossaryentry{latex}{
                name={LaTeX},
                description={ferramenta de computador para autoria de
                documentos criada por D. E. Knuth} }

\newglossaryentry{abntex2}{
                name={abnTeX2},
                see=latex,
                description={suíte para LaTeX que atende os requisitos das
                normas da ABNT para elaboração de documentos técnicos e científicos brasileiros} }