% ------------------------------------------------------------------------------------
% 2018-11-29 - Emerson Ribeiro de Mello - mello@ifsc.edu.br
% ------------------------------------------------------------------------------------
\documentclass[11pt]{article}
\usepackage[a4paper,hmargin=2.2cm,top=2cm,bottom=2cm]{geometry}
\usepackage{estilo-relatorio-article}
\usepackage[alf]{abntex2cite} % Citações padrão ABNT

% Esse pacote só foi usado para gerar blocos de textos "lorem ipsum" para preencher o exemplo.
\usepackage{lipsum}

\begin{document}


% ------------------------------------------------------------------------------------
% Capa
% ------------------------------------------------------------------------------------
\pagestyle{empty}
\begin{center}

% Logotipo do órgão
\includegraphics[scale=.7]{figuras/ifsc-logo-v}
\vspace{7.5cm}

% Título
{\huge \bfseries Relatório de Análise de Requisitos}

\vspace{.5cm}

% Subtítulo
{\large \bfseries Um modelo \LaTeX que faz uso da classe \texttt{article}}

\vfill
\end{center}

% Autor(es)
{\noindent \large \bfseries 
Emerson Ribeiro de Mello 
\\ Segundo Autor do Trabalho
}

% Data que o relatório foi gerado
\begin{flushright}
29 de novembro de 2018
\end{flushright}

\clearpage
\pagestyle{plain}
% ------------------------------------------------------------------------------------


% ------------------------------------------------------------------------------------
% Adicionando sumário
% ------------------------------------------------------------------------------------
\tableofcontents
\clearpage




% ------------------------------------------------------------------------------------
% Início da parte textual
% ------------------------------------------------------------------------------------

\section{Introdução}
\label{sec:introducao}

Na introdução é apresentada a organização do documento e uma breve descrição de como o mesmo foi construído.   Esse é um modelo em \LaTeX \cite{lamport94} e o estilo das referências bibliográficas segue as normas da ABNT, implementadas pelo pacote abnTeX2. A \autoref{fig:logolatex} apresentada o logo do \LaTeX.

\begin{figure}[ht]
    \centering
    \includegraphics[width=.5\linewidth]{figuras/latex-logo}
    \caption{Logo do \LaTeX}
    \label{fig:logolatex}
\end{figure}

\lipsum[1]

O restante do documento está organizado da seguinte forma. Na \autoref{sec:escopo} é apresentado o escopo do projeto. Na \autoref{sec:requisitos} são listados os requisitos do sistema.

\subsection{Escopo do projeto}
\label{sec:escopo}

Indica o propósito do sistema a ser desenvolvido. Por exemplo, descreve a necessidade que foi apresentada pela área requisitante. \lipsum[2]


\section{Requisitos do sistema}
\label{sec:requisitos}

Nessa seção deve ser apresentada a lista de pessoas que participaram do levantamento de requisitos. Pode-se basear nos livro de \cite{bezerra02}.

\lipsum[2]


\subsection{Requisitos funcionais}
\label{sec:reqfuncionais}

\begin{enumerate}
	\item O sistema deve permitir que alunos visualizem as notas obtidas
	\item O sistema deve permitir que os alunos façam matrícula nas disciplinas do semestre letivo
	\item O sistema deve permitir que os alunos possam obter seus históricos escolares
\end{enumerate}

\subsection{Requisitos não funcionais}
\label{sec:reqnaofuncionais}

\begin{enumerate}
	\item A interface do usuário deve ser simplificada e evitar múltiplos cliques (passos) para chegar em qualquer funcionalidade
	\item Fazer uso de protocolos criptográficos para troca de mensagens
	\item O sistema deverá ser desenvolvido em uma linguagem para web e que não dependa de plugins para o navegador
	\item O banco de dados precisar ser o MySQL
\end{enumerate}

\subsection{Regras de negócio}
\label{sec:regrasdenegocio}

\begin{enumerate}
	\item \textbf{Número máximo de matrículas por semestre letivo}
	\begin{itemize}
		\item Em um semestre letivo um aluno não poderá se matricular em um número de disciplinas cujo o soma de créditos ultrapasse 18.
	\end{itemize}
	\item \textbf{Número máximo de alunos por turma}
	\begin{itemize}
		\item Uma oferta de disciplina não pode ter mais de 40 vagas para matrícula.
	\end{itemize}
	\item \textbf{Validação de créditos}
	\begin{itemize}
		\item Um aluno só poderá solicitar validação de uma disciplina caso nunca tenha se matriculado na mesma.
	\end{itemize}
\end{enumerate}








% ----------------------------------------------------------
% Referências bibliográficas
% ----------------------------------------------------------
\bibliography{referencias}





\end{document}

