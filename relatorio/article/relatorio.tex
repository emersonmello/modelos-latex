% ------------------------------------------------------------%
% 2020-12-05
% 2015-2020 - Emerson Ribeiro de Mello - mello@ifsc.edu.br
% ------------------------------------------------------------%
\documentclass[11pt]{../../classes/ifscarticle}
\usepackage{../../classes/ifscutils}
\usepackage[alf]{abntex2cite} % Citações padrão ABNT

\renewcommand*{\theusecase}{UC-\thesection.\arabic{usecase}}


\usepackage{lipsum}

\AtBeginDocument{\thispagestyle{empty}}
\begin{document}
% ------------------------------------------------------------%
% Capa
% ------------------------------------------------------------%
\begin{center}

% Logotipo do órgão
\includegraphics[scale=.7]{../../classes/imagens/ifsc-v}
\vspace{7.5cm}

% Título
{\huge \bfseries Relatório de Análise de Requisitos}

\vspace{.5cm}

% Subtítulo
{\large \bfseries Um modelo \LaTeX~que faz uso da classe \texttt{article}}

\vfill
\end{center}

% Autor(es)
{\noindent \large \bfseries 
Emerson Ribeiro de Mello%
\\[.5em] Segundo Autor do Trabalho%
}

% Data que o relatório foi gerado
\begin{flushright}
05 de dezembro de 2020
\end{flushright}

\clearpage
\pagestyle{firstpage}
% ------------------------------------------------------------%

% ------------------------------------------------------------%
% Adicionando sumário
% ------------------------------------------------------------%
\tableofcontents
\clearpage

% ------------------------------------------------------------%
% Início do documento
% ------------------------------------------------------------%

\section{Introdução}
\label{sec:introducao}

Na introdução é apresentada a organização do documento e uma breve descrição de como o mesmo foi construído.   Esse é um modelo em \LaTeX~ \cite{lamport94} e o estilo das referências bibliográficas segue as normas da ABNT, implementadas pelo pacote abnTeX2. A \autoref{fig:logolatex} apresentada o logo do \LaTeX.

\begin{figure}[ht]
    \centering
    \includegraphics[width=.5\linewidth]{figuras/latex-logo}
    \caption{Logo do \LaTeX}
    \label{fig:logolatex}
\end{figure}

\lipsum[1]

O restante do documento está organizado da seguinte forma. Na \autoref{sec:escopo} é apresentado o escopo do projeto. Na \autoref{sec:requisitos} são listados os requisitos do sistema.

\subsection{Escopo do projeto}
\label{sec:escopo}

Indica o propósito do sistema a ser desenvolvido. Por exemplo, descreve a necessidade que foi apresentada pela área requisitante. \lipsum[2]


\section{Requisitos do sistema}
\label{sec:requisitos}

Nessa seção deve ser apresentada a lista de pessoas que participaram do levantamento de requisitos. Pode-se basear nos livro de \cite{bezerra02}.

\lipsum[2]


\subsection{Requisitos funcionais}
\label{sec:reqfuncionais}

\begin{enumerate}
    \item O sistema deve permitir que alunos visualizem as notas obtidas
    \item O sistema deve permitir que os alunos façam matrícula nas disciplinas do semestre letivo
    \item O sistema deve permitir que os alunos possam obter seus históricos escolares
\end{enumerate}

\subsection{Requisitos não funcionais}
\label{sec:reqnaofuncionais}

\begin{enumerate}
    \item A interface do usuário deve ser simplificada e evitar múltiplos cliques (passos) para chegar em qualquer funcionalidade
    \item Fazer uso de protocolos criptográficos para troca de mensagens
    \item O sistema deverá ser desenvolvido em uma linguagem para web e que não dependa de plugins para o navegador
    \item O banco de dados precisar ser o MySQL
\end{enumerate}

\subsection{Regras de negócio}
\label{sec:regrasdenegocio}

\begin{enumerate}
    \item \textbf{Número máximo de matrículas por semestre letivo}
    \begin{itemize}
        \item Em um semestre letivo um aluno não poderá se matricular em um número de disciplinas cujo o soma de créditos ultrapasse 18.
    \end{itemize}
    \item \textbf{Número máximo de alunos por turma}
    \begin{itemize}
        \item Uma oferta de disciplina não pode ter mais de 40 vagas para matrícula.
    \end{itemize}
    \item \textbf{Validação de créditos}
    \begin{itemize}
        \item Um aluno só poderá solicitar validação de uma disciplina caso nunca tenha se matriculado na mesma.
    \end{itemize}
\end{enumerate}


\section{Casos de uso}\label{sec:casos_de_uso}


\begin{usecase}{Consultar informações}{uc:consultarinfos}{Esse caso de uso descreve as etapas para o PAF obter informações como versão do SB, resumo criptográfico do SB, IdDAF, modelo, fabricante, valor atual do contador, certificado armazenado, estado do DAF, algoritmos criptográficos que o DAF provê suporte e identificadores dos documentos retidos na MT.}
    \begin{cabecalhoUC}
    \atorPrimario{PAF}
    % \atoresSecundarios{ABC, DEF}
    \precondicoesUC{DAF deve estar no estado \textsc{Pronto}, \textsc{Inativo} ou \textsc{Bloqueado}}
    \end{cabecalhoUC}

    \begin{fluxoprincipal}
        \item O PAF solicita ao DAF suas informações 
        \item O DAF retorna para o PAF o documento estruturado com suas informações
    \end{fluxoprincipal}  

    \begin{fluxoexcecao}{Pedido mal formado}
        \item O DAF retorna para o PAF uma mensagem de erro informando que o pedido foi mal formado)
    \end{fluxoexcecao}   
\end{usecase}


% ----------------------------------------------------------
% Referências bibliográficas
% ----------------------------------------------------------
\bibliography{referencias}


\end{document}

