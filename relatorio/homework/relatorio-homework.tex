% ------------------------------------------------------------%
% 2020-12-05
% 2015-2020 - Emerson Ribeiro de Mello - mello@ifsc.edu.br
% ------------------------------------------------------------%
\documentclass[11pt]{../../classes/ifscarticle}
\usepackage{../../classes/ifscutils}
\usepackage[alf]{abntex2cite} % Citações padrão ABNT

\renewcommand*{\theusecase}{UC-\thesection.\arabic{usecase}}


\usepackage{lipsum}

\AtBeginDocument{\thispagestyle{empty}}
\begin{document}
% ------------------------------------------------------------%
% Capa
% ------------------------------------------------------------%
\begin{titlepage}
\begin{center}
% Logotipo do IFSC
\includegraphics[scale=.7]{../../classes/imagens/ifsc-v}
\vspace{4cm}

\rule{\linewidth}{0.5pt} \\[6pt] 

% Título
{\huge \bfseries Laboratório 01}\\[3pt] 

% Subtítulo
{\large  Método construtor e modificadores de acesso}\\

\rule{\linewidth}{2pt}  \\[10pt]
\vspace{1cm}
\end{center}

\begin{minipage}{.9\linewidth}

% Nome do curso, disciplina e professor.
\begin{minipage}{0.15\textwidth}
\begin{flushleft} \large
\textbf{Curso:}\\
\textbf{Disciplina:}\\
\textbf{Professor:}
\end{flushleft}
\end{minipage}~
\begin{minipage}{0.8\textwidth}
\begin{flushleft} \large
Engenharia de Telecomunicações\\
POO29004 -- Programação Orientada a Objetos\\
Emerson Ribeiro de Mello
\end{flushleft}
\end{minipage}
\end{minipage}
\vspace{3cm}

% Nome dos alunos
\begin{minipage}{.9\linewidth}
\begin{flushright}
\textbf{Alunos}\\
Nome do primeiro aluno\\
Nome do segundo aluno
\end{flushright}
\end{minipage}
\vfill


% Coloque aqui a data
\begin{center}
05 de dezembro de 2020
\end{center}

\end{titlepage}
\pagestyle{firstpage}
% ------------------------------------------------------------%

% ------------------------------------------------------------%
% Para adicionar sumário
% ------------------------------------------------------------%
% \tableofcontents
% \clearpage
% ------------------------------------------------------------%


% ------------------------------------------------------------%
% Início do documento
% ------------------------------------------------------------%
\section{Introdução}
\label{sec:introducao}

\lipsum[1]

\begin{figure}[ht]
    \centering
    \includegraphics[width=.3\linewidth]{figuras/latex-logo}
    \caption{Logo do \LaTeX}
    \label{fig:logolatex}
\end{figure}

Na introdução é apresentada a organização do documento e uma breve descrição de como o mesmo foi construído.   Esse é um modelo em \LaTeX~ \cite{lamport94} e o estilo das referências bibliográficas segue as normas da ABNT, implementadas pelo pacote abnTeX2. A \autoref{fig:logolatex} apresentada o logo do \LaTeX. O restante do documento está organizado da seguinte forma. Na \autoref{sec:construtorjava} é apresentado o código de como fazer um método construtor em Java. Na \autoref{sec:construtorpython} é apresentado como fazer um método construtor em Python.

\subsection{Método construtor em Java}
\label{sec:construtorjava}


\begin{lstlisting}[language=java]
public class Carro{
    private String modelo = "Fusca";
    private int velocidade = 0;
    
    public Carro(String mo, int ve){
        modelo = mo;
        velocidade = ve;
    }
}
\end{lstlisting}


\subsection{Método construtor em Python}
\label{sec:construtorpython}


\begin{lstlisting}[language=python]
class Carro:
    def __init__(self, modelo='Fusca', velocidade=0):
        self.modelo = modelo
        self.velocidade = velocidade
\end{lstlisting}






% ----------------------------------------------------------
% Referências bibliográficas
% ----------------------------------------------------------
\bibliography{referencias}





\end{document}

