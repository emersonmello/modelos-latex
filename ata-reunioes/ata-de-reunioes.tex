% ------------------------------------------------------------%
% 2020-12-05
% 2015-2025 - Emerson Ribeiro de Mello - mello@ifsc.edu.br
% ------------------------------------------------------------%
\documentclass[11pt]{ata}

\usepackage[brazilian]{babel}
\usepackage{url,graphicx,pgf}
\urlstyle{sf}


\usepackage{lipsum}  % produces dummy text


%%%%%%%%%%%%%%%%%%%%%%%%%%%%%%%%%%%%%%%%%%%%%%%%
% Set Helvetica Font in Text and Math in LaTeX %
%%%%%%%%%%%%%%%%%%%%%%%%%%%%%%%%%%%%%%%%%%%%%%%%
\renewcommand{\familydefault}{\sfdefault}
\usepackage[scaled=1]{helvet}
\usepackage[helvet]{sfmath}
\everymath={\sf}
\usepackage[T1]{fontenc}




%-----------------------------------------------------------
% Informações que aparecerão no cabeçalho
%-----------------------------------------------------------
% Logo
% \pgfdeclareimage[width=5.5cm]{logo}{imagens/ifsc-h-sje.png}
\pgfdeclareimage[width=1.3\linewidth]{logo}{imagens/ifsc-header.png}


% \instituicao{Ministério da Educação\\
% Secretaria de Educação Profissional e Tecnológica\\
% Instituto Federal de Educação, Ciência e Tecnologia de Santa Catarina\\
% Campus São José}

% \departamento{Área de Telecomunicações}
\departamento{Colegiado do curso da Engenharia de Telecomunicações}


%-----------------------------------------------------------
% Informações que devem ser preenchidas para cada reunião
%-----------------------------------------------------------
% Formas de uso do comando \participante

% Para presentes na reunião
% \participante{Nome do participante}

% Para ausentes sem justificativa
% \participante{Nome do participante}[ausente]

% Para ausentes com justificativa
% \participante{Nome do participante}[ausente][informar a justificativa]
%-----------------------------------------------------------


\participante{Emerson Ribeiro de Mello}

\participante{Fulana dos Santos}[ausente]

\participante{Fulano da Silva}

\participante{\papel{Juca Falante}{Comentarista}}

\participante{Juca Luca}[ausente][capacitação]

\participante{\presidente{Odilson Tadeu}}

\participante{\secretario{Ricardo Escrev}}

\participante{\secretaria{Maria Escriba}}

\participante{\titular{Roseana Parti}}

\participante{Rosevânia Nomera}[ausente][em aula]

\participante{\suplente{Jucênica Virtu}}


%-------------------
% Se houver convidados descomente a lista abaixo
%-------------------
% \listaconvidados{
% Professora Tal de Tal,
% \papel{Seo Joca}{coordenador de pesquisa}}

\date{03 de dezembro de 2019}

%-----------------------------------------------------------
% Descomente a linha abaixo para que os trechos protegidos 
% por \censor e \blackout não sejam mais hachurados.
%-----------------------------------------------------------
% \StopCensoring

\begin{document}

\begin{reuniao}{Ordinária} % Extraordinária

%-----------------------------------------------------------
% Lista de informes
%-----------------------------------------------------------
\begin{informes}
\item \textbf{Como fazer uma ata de reunião informal}. Aspecto informal que se caracteriza como uma espécie de súmula na qual serão registradas apenas as informações mais relevantes e que posteriormente poderá ser enviado ao RH. Ex: reuniões departamentais, reuniões com clientes e entre outros. Principais tópicos que devem constar na ata simples:
\begin{itemize}
    \item \textbf{Data, local, horário de início e fim da reunião}: é necessário saber onde e quando as pautas foram discutidas;
    \item \textbf{Pessoas presentes e seus cargos}: é preciso informar por quem as decisões foram tomadas;
    \item \textbf{Pauta da reunião}: uma das principais informações que deve constar na ata a fim de saber o propósito para o qual as pessoas se reuniram;
    \item \textbf{Discussões abordadas}: parte fundamental para registro das discussões para esclarecer porque as decisões foram tomadas e porque algumas ideias foram abandonadas a fim de evitar que a mesma discussão ocorra várias outras vezes;
    \item \textbf{Registro das decisões}: outra parte fundamental para registrar o que de fato foi acordado entre os integrantes e listar quais serão os próximos passos;
    \item \textbf{Compromissos}: essencial para registrar os prazos para execução de cada tarefa e é onde se estabelece o dia, horário, local e participantes da próxima reunião;
\end{itemize}
Uma boa ata irá conter esses tópicos. Dificilmente uma ata conterá mais informações além destas. É importante lembrar que a ata deve ser um documento sucinto, de fácil leitura e identificação (em especial) das decisões tomadas. E uma ata que não abranja todos estes itens ficará incompleta, deixando os leitores com dúvidas sobre algum aspecto da reunião. Esse texto foi extraído na íntegra do \textit{site} \url{https://manualdasecretaria.com.br/ata-de-reuniao-como-fazer/} em 03 de maio de 2019.

\item A viagem do \censor{Fulano de Tal} não acontecerá.
\end{informes}

%-----------------------------------------------------------
% Lista dos assuntos tratados (pontos de pauta)
%-----------------------------------------------------------
\begin{assuntos}
\item 
\aprovada

\item
\lipsum[2]
    
\item
\lipsum[4]

\item
\lipsum[4-5]
\end{assuntos}
\end{reuniao}

% Opcional
\proximareuniao{10 de outubro de 2019.}


\end{document}

