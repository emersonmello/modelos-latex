% ------------------------------------------------------------%
% 2020-12-05
% 2015-2020 - Emerson Ribeiro de Mello - mello@ifsc.edu.br
% ------------------------------------------------------------%
\documentclass[11pt]{../classes/ifscarticle}
\usepackage{../classes/ifscutils}
\geometry{a4paper,hmargin=2cm,top=3.5cm,bottom=2cm,headheight=3cm,heightrounded}


\fancypagestyle{followingpage}{
	\fancyhead[C]{\pgfuseimage{cabecalho}}
}
\pgfdeclareimage[width=1.29\linewidth]{cabecalho}{../classes/imagens/ifsc-header.png}

%-----------------------------%
% Citações padrão ABNT com biblatex
%-----------------------------%
\usepackage[style=abnt,noslsn,justify]{biblatex}
\usepackage{csquotes}

%-----------------------------%
% Arquivo com a bibliografia
%-----------------------------%
\addbibresource{referencias-poo.bib}


\begin{document}

\begin{center}
	{\huge Plano de ensino}\vspace*{.5cm}
\end{center}

% \section{Dados da disciplina}
\noindent\fbox{%
\begin{minipage}{.98\textwidth}%
	\begin{small}
	\begin{tabular}{@{}l l@{}}
	\textbf{Curso}                & Engenharia de Telecomunicações\\
	\textbf{Unidade curricular}   & Programação Orientada a Objetos\\
	\textbf{Semestre}             & 2024-01\\
	\textbf{Carga horária}        & 80 horas\\
	\textbf{Professor}            & Emerson Ribeiro de Mello\\
	\textbf{Página da disciplina} & \url{https://emersonmello.me}\\
	\end{tabular}
	\end{small}			
\end{minipage}%
}\vspace*{.5cm}

\section{Ementa}

Introdução ao paradigma da orientação a objetos: Classes, objeto, associações entre classes, herança. Introdução à linguagem de modelagem unificada (UML): Diagramas de caso de uso, classes, sequência. Introdução a linguagem de programação Java: Tipos de dados primitivos, estruturas de controle, vetores; concepção de projeto orientado a objetos, herança, polimorfismo; interfaces gráficas amigáveis.

\section{Objetivos}

Ao término da disciplina o aluno será capaz de modelar, implementar e testar software de média complexidade na linguagem Java e de acordo com o paradigma da programação orientada a objetos. Os objetivos específicos da disciplina são:

\begin{itemize}
    \item Introduzir os conceitos da programação orientada a objetos;
    \item Apresentar a linguagem de programação Java e a linguagem de modelagem unificada (UML);
    \item Usar de forma efetiva ferramentas como ambiente integrado de desenvolvimento e sistema de controle de versão para trabalhar de forma colaborativa;
    \item Modelar software de média complexidade por meio de diagramas UML comportamentais e estruturais.
\end{itemize}



\section{Metodologia}

Essa unidade curricular será ministrada com aulas expositivas-dialogadas e práticas em laboratório sob a supervisão do professor. As atividades empregarão a metodologia de aprendizado baseado em projetos, em que desafios são lançados e o docente orienta os estudantes em suas soluções com base nos conceitos sendo estudados. As atividades práticas serão conduzidas em computadores com o sistema operacional Linux e com os softwares aplicativos para automatização de projetos e o kit de desenvolvimento Java, além de aplicações web para desenvolvimento de diagramas UML. As aulas práticas serão conduzidas nos laboratórios de informática do campus.

\section{Conteúdo programático}

\begin{enumerate}
	\item Introdução à programação orientada a objetos
	\item Introdução à linguagem de modelagem unificada (UML)
	\item Introdução à linguagem de programação Java
	\item Concepção de projeto orientado a objetos
	\item Herança
	\item Polimorfismo
\end{enumerate}

% -----------------------------
% Referências bibliográficas
% -----------------------------

\nocite{*}

% Vai imprimir todas as referências do arquivo referencias-poo.bib que possuam a palavra-chave "basica" 
\printbibliography[heading=bibintoc,keyword={basica},title={Bibliografia básica}]%

% Vai imprimir todas as referências do arquivo referencias-poo.bib que possuam a palavra-chave "complementar"
\printbibliography[heading=bibintoc,keyword={complementar},title={Bibliografia complementar}]%
% -----------------------------




\end{document}