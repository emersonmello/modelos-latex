% ------------------------------------------------------------%
% 2015-2021 - Emerson Ribeiro de Mello <mello@ifsc.edu.br>
% ------------------------------------------------------------%
% Sets aspect ratio to 4:3, and frame size to 128 mm by 96 mm
\documentclass{beamer}
% Sets aspect ratio to 16:9, and frame size to 160 mm by 90 mm.
% \documentclass[aspectratio=169]{beamer}
% Sets aspect ratio to 16:10, and frame size to 160 mm by 100 mm.
% \documentclass[aspectratio=1610]{beamer}

% -------------------------------------------------%
%  Package options
% -------------------------------------------------%
% 
% textbgcolor   - frametitle background color. default: 0d4f4d
% textfgcolor   - frametitle foreground color. default: ffffff
% slidebgcolor  - slide background color. default: eef1ec
% slidefgcolor  - slide text foreground color. default: 000000
% authorfgcolor - author, institute and date color. default: 000000
% itemsep - space between items (itemize, enumerate). default: 7pt
\usepackage[textbgcolor=0d4f4d,textfgcolor=ffffff,itemsep=7pt]{beamerthemeifscyan}
% -------------------------------------------------%

% A good place to get some colors
% https://material.io/resources/color/#!/?view.left=0&view.right=0&primary.color=0d4f4d
% cyan: #0D4F4D, light: #417b79,  dark: #002625
% IFSC green: normal: #32A041, light: #69d26f, dark: #007013
% IFSC red: normal: #C8191E, light: #ff5747, dark: #8f0000 
% Other colors for textbgcolor
% purple 4527a0, blue 0d47a1, grey 546e7a, redwine 880e4f, brown 6d4c41, yelllow (bg=#fbc02d, fg=000000)

% Logo
\pgfdeclareimage[height=.4\paperheight]{ifsclogo}{img/ifsclogo.pdf}


% -------------------------------------------------%
%              Título 
% -------------------------------------------------%
\title{Tema IFSCyan para~\LaTeX~Beamer}
\subtitle{Nova organização para possibilitar personalizações}

\author{Prof. Emerson Ribeiro de Mello, Dr.}
\institute{%
\url{mello@ifsc.edu.br}%
}%
\date{16/07/2021}
% -------------------------------------------------%



% -------------------------------------------------%
%  Início do documento 
% -------------------------------------------------%
\begin{document}

\begin{frame}[plain]
    \titlepage
\end{frame}

\begin{frame}[plain, noframenumbering]{Licenciamento}
    \licenciamentoLivre
\end{frame}

\begin{frame}[plain, noframenumbering]{Sumário}
    \tableofcontents
\end{frame}



\section{Listas}


\begin{frame}{Lista de itens}
    \begin{enumerate}
        \item Primeiro item
        \begin{itemize}
            \item \Myhref{https://emersonmello.me}{Aqui tem um exemplo de \textit{link} para um \textit{site} usando o comando \texttt{Myhref}}
            \item Segundo item
        \end{itemize}
        \item Segundo item
        \item Terceiro item 
        \begin{itemize}
            \item Neste linha é apresentado um item que tem um texto grande que deverá ocupar mais de uma linha no slide. Assim, espera-se mostrar como será o alinhamento na margem esquerda
            \item Segundo item
        \end{itemize}
    \end{enumerate}
\end{frame}

\subsection{Animações}

\begin{frame}{Animação com lista de itens}
\begin{itemize}[<+->]
    \item Primeiro item
    \item Segundo item
    \item Terceiro item
\end{itemize}
\end{frame}

\begin{frame}{Uso do \texttt{alert}}
    \begin{itemize}
        \item<1-| alert@1> Aparecerá desde do slide 1 e receberá destaque no slide 1
        \item<1-| alert@2> Aparecerá desde do slide 1 e receberá destaque no slide 2
        \item<1-| alert@3> Aparecerá desde do slide 1 e receberá destaque no slide 3
        \item<4-| alert@4> Aparecerá desde do slide 4 e receberá destaque no slide 4
    \end{itemize} 
\end{frame}

\subsection{Colunas}

\begin{frame}{Listas em colunas}
    \begin{columns}[t, onlytextwidth]
        \column{0.5\textwidth}
            \begin{itemize}
                \item Item 1
                \begin{itemize}
                    \item Subitem 1.1
                    \item Subitem 1.2
                \end{itemize}
                \item Item 2
                \item Item 3
            \end{itemize}
        
        \column{0.5\textwidth}
            \begin{enumerate}
                \item First
                \item Second
                \begin{enumerate}
                    \item Sub-first
                    \item Sub-second
                \end{enumerate}
                \item Third
            \end{enumerate}
    \end{columns}
\end{frame}

\begin{frame}{Listas e figuras}
\begin{columns}
    \column{.5\linewidth}
     \begin{center}
        \includegraphics[width=\linewidth]{figs/4x3.png}
     \end{center}
    \column{.5\linewidth}
    \begin{itemize}
        \item Item 1
        \begin{itemize}
            \item Subitem 1.1
            \item Subitem 1.2
        \end{itemize}
        \item Item 2
        \item Item 3
    \end{itemize}
\end{columns} 
\end{frame}


\section{Blocos}


\begin{frame}{Blocos}
    \begin{block}{Esse é um bloco}
        Isso é um teste
    \end{block}
    \begin{block}{}
    Bloco sem título	
    \end{block}
    \begin{alertblock}{Alerta}
        Esse é um alerta
    \end{alertblock}
\end{frame}


\begin{frame}{Blocos personalizados}

    \begin{informacao}
    Isso é uma mensagem de informação
    \end{informacao}
    
    \begin{informacaoazul}
        Isso é uma mensagem de informação
    \end{informacaoazul}

    \begin{atencao}
        Isso é uma mensagem de atenção
    \end{atencao}
    
    \begin{cuidado}
        Isso é uma mensagem de cuidado
    \end{cuidado}
\end{frame}

\begin{frame}{Horários}
    {Veja outros ícones em \url{https://ctan.org/pkg/fontawesome}}
    \begin{itemize}
        \item Horário das aulas 
        \begin{caixa}[azul]{white}{\faCalendar}
            \begin{itemize}
                \item 15:40 -- 17:30 - quarta-feira
                \item 13:30 -- 15:20 - quinta-feira
            \end{itemize}
        \end{caixa}
        \item Atendimento extraclasse
        \begin{caixa}[azul]{white}{\faCalendar}
            \begin{itemize}
                \item 10:00 -- 12:00 - sexta-feira
            \end{itemize}
        \end{caixa}
    \end{itemize}
\end{frame}


\section{Figuras}

\begin{frame}{Figura 4x3}
    \begin{center}
        \includegraphics[height=.8\paperheight]{figs/4x3.png}
    \end{center}
\end{frame}
    

\begin{frame}{Figura 16x9}
\begin{center}
    \includegraphics[width=.95\linewidth]{figs/16x9.png}
\end{center}
\end{frame}

\section{Tabelas}

\begin{frame}{Tabelas}
\begin{center}
    \begin{tabular}{l r r}\\\toprule
        Produto & Quantidade & Valor (R\$) \\\midrule
        Água    &      100   &   1,50 \\ 
        Banana  &       10   &   3,00 \\
        Maça    &        1   &   2,50 \\ \bottomrule
    \end{tabular}
\end{center}
\end{frame}

\section{Incluindo código fonte}


\begin{frame}[fragile]{Olá mundo em C}
    \lstinputlisting[style=ansic,caption={Olá mundo},label={cod:olac},captionpos=t]{codigos/ola.c}
\end{frame}

\begin{frame}[fragile]{Olá mundo em Java}
\begin{lstlisting}[style=java]
public class App{

    public static void main(String[] args){
        System.out.println("Olá mundo");
    }

}
\end{lstlisting}
\end{frame}

\begin{frame}{Referências}
    \nocite{*}
    \bibliography{demo_bibliography}
    \bibliographystyle{plain}
\end{frame}

\appendix

\section{Slides de backup}

\begin{frame}{Slide de backup}
    \usebeamercolor[fg]{normal text}
    Slides de backup são úteis para incluir materiais adicionais, necessários somente para ajudar a responder possíveis perguntas da plateia
    \vfill
    O pacote \texttt{appendixnumberbeamer} é usado para não numerar os slides de backup
\end{frame}


\end{document}